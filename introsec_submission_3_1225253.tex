\documentclass{scrartcl}

\usepackage[utf8]{inputenc}
\usepackage[T1]{fontenc}
\usepackage[ngerman, english]{babel}
\selectlanguage{english}
\usepackage{listings}
\usepackage{hyperref}
\usepackage{upquote}

\lstset{ %
	frame=single,
	numbers=left,
	breaklines=true,
	breakautoindent=false,
	breakatwhitespace=true,
	% keepspaces=true,
	basicstyle=\ttfamily,
	upquote=true,
}

\newcommand\solution[2]{{\paragraph{#1}#2}}
\newcommand\todo[1]{TODO: #1}


\title{Assignment 3 - SQL Injection}
\author{$<Simon>$ $<Rommer>$, $<1225253>$}
\date\today{}

\begin{document}

\maketitle

\begin{abstract}
	
A friend of yours has asked you to perform a security audit on the administration of members in her golf club. She asks you to test if there are any security issues concerning the MySQL database implementation. She gives you a short introduction to the system, but leaves you to find out other information you need for your test:

\end{abstract}



\section{SQL Injection}

\section*{Assignment}
Be aware that SQL injection can be a cumbersome task and it may take a while
until you find the right query. Therefore it is advisable to start the
assignment early and get back to it after a while when you find yourself stuck.
You might want to read a bit into the syntax of SQL and search for information
on SQL injection. In this assignment you will exercise what is called a ``Blind
SQL injection'' which means you will not get any error messages from the server
if the query you passed is wrong or doesn't yield any results. Note that in
this exercise no output can mean you are on a good way. 



\section*{Level 1}
\subsection*{Assignment}
Try to login without having any user data.

\subsection*{Description}
\solution{Username}{asdf}
\solution{PIN}{' or '1' = '1} 

I used the escape character $'$ to escape the input query and be able to type in something that was interpreted as actual code. \\
The most simple SQL-Injection is $ ' or '1' = '1 $ which means in the context of the login prompt that the login was successful if the right user was found or $1=1$ is true \footnote{\url{https://www.youtube.com/watch?v=FwIUkAwKzG8}}. This is always the case so the login was successful. I tried out different combination, but it seemed that the Username field was propperly escaped hence the PIN-field was vulnerable.\\
After that I was logged in as the first user in the  database \footnote{\url{https://www.youtube.com/watch?v=h-9rHTLHJTY}}.

\section*{Level 2}
\subsection*{Assignment}
Find out which of the members has the highest balance on his/her account. You will not be able to see the balance on the website, you must find it out by passing an appropriate SQL query to the server.

\subsection*{Description}
\solution{Name}{Member with highest balance}
\solution{Balance}{Balance of said member}

For the second part of the assignment I started to dig deeper into tutorials about SQL-Injections\footnote{\url{http://www.kalitutorials.net/2014/03/sql-injection-how-it-works.html}}\footnote{\url{http://www.kalitutorials.net/2014/03/hacking-websites-using-sql-injection.html}}\footnote{\url{http://www.kalitutorials.net/2015/02/blind-sql-injection.html}}\footnote{\url{http://www.kalitutorials.net/2014/03/hacking-website-with-sqlmap-in-kali.html}}

\section*{Level 3}
\subsection*{Assignment}
There is a members database which consists of two tables \texttt{regular} and
\texttt{vip}. Find out the \texttt{memberno} of the member who had the highest
balance in step two. The \texttt{name} of every member has a suffix
\texttt{(reg)} or \texttt{(VIP)} - this way you will recognize which table you are operating on. Again, you will
not be able to see the \texttt{memberno} on the website but you must try to find it by
using an appropriate SQL query.

\subsection*{Description}
\solution{MemberNo}{MemberNo with highest balance}

\todo{What is the \texttt{memberno} of the member with the highest balance from the step two? Was he/she a regular or a VIP member? What query (or queries) did you use to accomplish this task? Which table have you been operating on all along?} \\


\end{document}

